\documentclass[11pt,a4wide]{article}
\usepackage{verbatim}
\usepackage{listings}
\usepackage{graphicx}
\usepackage{a4wide}
\usepackage{color}
\usepackage[options]{SIunits}
\usepackage{amsmath}
\usepackage{amssymb}
\usepackage[dvips]{epsfig}
\usepackage[utf8]{inputenc}
\usepackage[OT1]{fontenc}
\usepackage{cite} % [2,3,4] --> [2--4]
\usepackage{shadow}
\usepackage{hyperref}

\setcounter{tocdepth}{2}
%
\lstset{language=c++}
\lstset{alsolanguage=[90]Fortran}
\lstset{basicstyle=\small}
\lstset{backgroundcolor=\color{white}}
\lstset{frame=single}
\lstset{stringstyle=\ttfamily}
\lstset{keywordstyle=\color{red}\bfseries}
\lstset{commentstyle=\itshape\color{blue}}
\lstset{showspaces=false}
\lstset{showstringspaces=false}
\lstset{showtabs=false}
\lstset{breaklines}

\begin{document}
\title{report on project $2$}
\author{Ekaterina Ilin and Isabelle Gauger\\GitHub: \url{https://github.com/CPekaterina/project-no2}
}
\maketitle
\tableofcontents
\newpage

questions a)

We look at the equation

\begin{equation}
\left|t\right| = \left|-\tau\pm\sqrt{1+\tau^2}\right|
\end{equation}

and consider the three cases $\tau=0$, $\tau>0$ and $\tau<0$.

For $\tau=0$ we get

\begin{equation}
\left|t\right| = \left|\pm1\right| = 1
\end{equation}

For $\tau>0$ the absolute value of $t$ is smaller for the solution with plus.  

\begin{equation}
\left|t\right| = \left|-\tau+\sqrt{1+\tau^2}\right|
\end{equation}

Now we look at $\tau\rightarrow\infty$:   

\begin{equation}  
\lim\limits_{\tau \rightarrow \infty}{\left|t\right|}=\lim\limits_{\tau \rightarrow \infty}{\left|-\tau+\sqrt{1+\tau^2}\right|}=0  
\end{equation}

For $\tau<0$ the absolute value of $t$ is smaller for the solution with minus.

\begin{equation}
\left|t\right| = \left|-\tau-\sqrt{1+\tau^2}\right|
\end{equation}

So now we look at $\tau\rightarrow-\infty$:

\begin{equation}
\lim\limits_{\tau \rightarrow -\infty}{\left|t\right|}=\lim\limits_{\tau \rightarrow -\infty}{\left|-\tau-\sqrt{1+\tau^2}\right|}=0  
\end{equation}

We have seen that for $\tau=0$ the absolute value of $t$ is one and if we increase or decrease $\tau$ it approachs zero.

\begin{equation}
\left|\tan{\theta}\right|\leq1 \text{for} \left|\theta\right|\leq\frac{\pi}{4}
\end{equation}

So if we choose $t$ to be the smaller of the roots $\left|\theta\right|\leq\frac{\pi}{4}$ what is minimizing the difference between the matrices A and B. This can be seen if we look at the given equation

\begin{equation}
||{\bf B}-{\bf A}||_F^2=4(1-c)\sum_{i=1,i\ne k,l}^n(a_{ik}^2+a_{il}^2) +\frac{2a_{kl}^2}{c^2}
\end{equation}

$(1-c)$ at the beginning of the equation becomes zero when $\cos{\theta}=1$ and than the total first part of the equation is zero. Also the second part of the equation reaches its minimum value for $\cos{\theta}=1$. For $\left|\theta\right|\leq\frac{\pi}{4}$ the value of $\cos{\theta}$ is between $1$ and $\approx 0.7$ so the difference between the matrices A and B we get is near the minimum. This means that the non-diagonal matrix elements of A are nearly zero, what is what we want to achieve.  

  
 
     
\end{document}

